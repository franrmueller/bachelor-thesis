\section{Introduction}\label{sec:intro}
Organizations across industries continue to face persistent challenges in achieving operational excellence. Fragmented processes, manual interventions, and inconsistent data quality undermine efficiency and decision-making. Legacy workflows and siloed systems exacerbate these inefficiencies, while traditional automation approaches often lack the adaptability needed in dynamic business environments. For companies, this translates into slower response times, higher compliance risks, and limited scalability—issues that directly threaten competitiveness.

Generative artificial intelligence (GenAI) opens new possibilities to extend automation beyond deterministic scripts, enabling adaptive, tool-using agents that support governance, decision quality, and agility. However, despite this potential, organizations and the academic literature alike lack structured strategies and conceptual frameworks for embedding such agentic capabilities into operational workflows in a scalable, value-driven way. This gap motivates the present research.

This thesis investigates how multi-agent systems can serve as a reference architecture for integrating GenAI into enterprise workflow automation. The central research question is:

\vspace{0.5\baselineskip}
\emph{How can a multi-agent architecture be designed to integrate GenAI capabilities into workflow automation, in order to enhance agility, compliance, \& decision quality to achieve operational excellence?}
\vspace{0.5\baselineskip}

To answer this question, the study addresses the following sub-questions:
\begin{itemize}
    \item \emph{What are the strengths \& limitations of GenAI in the context of workflow automation using a multi-agent architecture?}
    \item \emph{Which design requirements \& agent roles are necessary to align a multi-agent architecture with the goals of operational excellence?}
    \item \emph{Under which conditions is deploying a generative multi-agent architecture justified over traditional automation approaches?}
\end{itemize}

Methodologically, the thesis applies Design Science Research (DSR) to develop a conceptual reference architecture. The approach synthesizes requirements from academic literature and operational excellence principles, models agent roles and interactions, and derives applicability conditions for real-world deployment.

The core contribution of this work is a conceptual design of a multi-agent system that leverages GenAI to support operational excellence in enterprise workflows. Specifically, it delivers:
\begin{enumerate}
    \item A structured synthesis of system requirements derived from academic literature and operational excellence principles.
    \item A conceptual architecture detailing agent roles, interactions, and integration points
    \item A set of applicability conditions and design considerations to guide future deployment and evaluation of generative multi-agent architectures in practice.
\end{enumerate}

The scope is limited to conceptual design; formal evaluation and technical implementation are proposed as future work. The approach remains industry-agnostic but draws illustrative examples from the financial services sector, given its regulatory complexity and reliance on legacy systems.

The thesis is structured as follows: Section 2 outlines the research methodology, including DSR and supporting methods. Section 3 presents a literature review on operational excellence, automation paradigms, and multi-agent systems. Section 4 develops applicability conditions and use case illustrations. Section 5 introduces the conceptual architecture design, and Section 6 concludes with reflections and directions for future research.

\section{Methodology}\label{sec:method}
This thesis applies the design science research methodology to create a conceptual artifact: a multi-agent architecture for workflow automation. Practically, the approach unfolded in three steps: (1) reviewing the literature on operational excellence, workflow automation, and agentic AI; (2) deriving and structuring requirements from literature and case material into a requirements model; and (3) designing a conceptual system architecture using System Modeling Language (SysML).

Supporting methods included Mayring-style qualitative content analysis (QCA) for the review, requirements engineering (RE) and systems analysis for the requirements model, and MBSE to structure the architecture and ensure requirement-to-design traceability. Within DSR, the work deliberately focuses on problem identification, objective definition, and conceptual design, while instantiation/demonstration and formal evaluation are out of scope given the bachelor-thesis format and resource constraints. This scoping maintains methodological rigor while keeping the contribution focused: a well-argued reference architecture ready for subsequent implementation and empirical evaluation.

\subsection{Qualitative Content Analysis}\label{subsec:qca}
To ensure a structured literature review, this thesis employed qualitative content analysis according to \textcite{mayringQualitativeContentAnalysis2022}. QCA provides a transparent, rule-based procedure for synthesizing knowledge from textual sources while retaining the interpretative dimension. Its use in this thesis supports the DSR process \parencite{peffersDesignScienceMethodology2007} in the problem identification and motivation phase, where the goal is to understand the state of the problem domain and justify the value of a solution.\\\\
\textbf{Analytical Units} \quad Following Mayring’s guidelines, the analytical framework was defined prior to coding:
\begin{itemize}
    \item \textit{Analysis unit}: the overall literature corpus addressing operational excellence, workflow automation, and agentic AI.
    \item \textit{Context unit}: individual publications (books, academic articles, industry reports and fairs, and case studies).
    \item \textit{Coding unit}: specific textual statements or conceptual claims relevant to the intersection of operational excellence, automation paradigms, and AI-based multi-agent systems.
\end{itemize}\\\\
\textbf{Category Development} \quad A \textit{mixed deductive–inductive approach} was used:
\begin{itemize}
    \item \textit{Deductive categories} were derived from established theory, including operational excellence dimensions (adaptability, compliance, decision quality) and prior automation frameworks (RPA, IPA).
    \item \textit{Inductive categories} were generated from the material itself, capturing emerging issues such as “guardrails,” “observability,” and “traceability” in agentic AI systems.
\end{itemize}

The coding rules followed Mayring’s principle of rule-governed categorization, ensuring consistency and avoiding arbitrary interpretation.\\\\
\textbf{Integration into the DSR Process} \quad The resulting categories served two methodological functions:
\begin{enumerate}
    \item \textit{Problem Representation}: Categories structured how the research problem was represented, aligning with \textcite{hevnerDesignScienceInformation2004}, who emphasize that effective constructs are essential to problem framing.
    \item \textit{Derivation of Objectives}: Categories were transformed into metarequirements that guided the definition of solution objectives in Activity 2 of the DSR methodology \parencite{peffersDesignScienceMethodology2007}.
\end{enumerate}

Thus, the QCA ensured that the literature review in Section~\ref{sec:lit-rev} went beyond descriptive reporting and produced a structured synthesis directly traceable to the design requirements of the conceptual artifact.

In practical terms, the QCA was applied by systematically coding the literature across the three thematic pillars of this thesis: operational excellence, automation paradigms, and multi-agent systems. Deductive categories were derived from existing theory, while inductive categories emerged during the coding process. For each publication, statements were assigned to categories according to predefined coding rules, enabling a structured synthesis of the material. The resulting categories serve as the foundation for the requirements engineering in Section~\ref{subsec:re\&sa} and structure the literature review in Section\ref{sec:lit-rev}.

\subsection{Requirements Engineering \& System Analysis}\label{subsec:re\&sa}

\subsection{Model-Based Systems Engineering}\label{subsec:mbse}

\section{Literature Review}\label{sec:lit-rev}

This section synthesizes academic and industry perspectives on three pillars underpinning this research: (i) the evolving definition of operational excellence, (ii) the shift in enterprise automation from rules-based RPA to goal-oriented generative agents, and (iii) the role of multi-agent systems in complex enterprise settings. The review follows a qualitative content analysis approach \parencite{mayringQualitativeContentAnalysis2022} and remains problem-oriented: each subsection surfaces implications for the requirements engineered in Section~2.2 and the architecture designed in Section~5.

\subsection{The Evolving Definition of Operational Excellence}
Operational excellence (OpEx) emerged from manufacturing disciplines—Lean and Six Sigma—where the primary levers were waste elimination, variance reduction, and standard work \parencite{Womack1990,Harry1998}. Contemporary excellence models broaden this focus to services and knowledge work, emphasizing culture, strategy alignment, governance, and learning. Representative references include the Shingo Model and the EFQM Model, both of which integrate people, process, and result dimensions while stressing continuous improvement and adaptability \parencite{Shingo2014,EFQM2020}.

In data-rich and highly regulated domains (e.g., finance), the OpEx constraint is increasingly cognitive rather than purely procedural: human attention, handoffs, and decision latency limit throughput and quality \parencite{Hammer2004,Davenport2018}. Yet most excellence frameworks still presuppose human-led improvement cycles, offering limited guidance on embedding autonomous, machine-driven capabilities for sustained optimization, monitoring, and compliance. This gap motivates evaluating whether agentic, tool-using AI can extend OpEx from periodic human interventions toward continuous, machine-assisted improvement and control. For this thesis, the implications are twofold: requirements should (1) include decision-quality, compliance, and auditability as first-class non-functional goals, and (2) anticipate organizational enablers (roles, governance) that allow AI systems to participate in improvement cycles.
\subsection{Automation Paradigms: From RPA to GenAI Agents}
Robotic Process Automation (RPA) has proven effective for stable, rule-based back-office tasks and has delivered measurable cost, speed, and quality benefits \parencite{Lacity2016}. However, RPA’s brittleness under interface or policy changes and its limited handling of unstructured inputs constrain scalability in dynamic environments \parencite{Syed2021}. ``Intelligent'' variants (often termed IPA) augment RPA with OCR, NLP, and ML components, broadening applicability but remaining predominantly workflow- and rule-driven.

GenAI agents mark a paradigm shift: powered by large language models and tool orchestration, agents can decompose goals into sub-tasks, call external tools/APIs, reflect on intermediate outputs, and adapt strategies during execution \parencite{Park2023,Rodriguez_Agents_2025}. Conceptually, this elevates automation from \emph{process mimicry} (predefined steps) to \emph{goal-oriented problem solving} (contextual reasoning under uncertainty). The literature highlights open challenges central to enterprise adoption—reliability under distribution shift, controllability, explainability, data protection, and governance of autonomous actions \parencite{Bommasani2022}. For system design, this motivates requirements for guardrails (policy constraints, approvals), observability (telemetry, traces), and life-cycle management (versioning, evaluation) that exceed what is typical for RPA.
\subsection{Multi-Agent Systems in Enterprise Contexts}
multi-agent system comprise autonomous, interacting agents that perceive, decide, and act within a shared environment \parencite{Wooldridge2009}. Multi-agent systems have long been proposed for enterprise-scale coordination problems—supply chains, scheduling, and distributed control—leveraging decomposition, local decision-making, and coordination protocols \parencite{Parunak1999,JenningsBussmann2003}. Established patterns include the blackboard architecture for shared problem solving \parencite{Nii1986} and market/negotiation mechanisms such as the Contract Net Protocol for dynamic task allocation \parencite{Smith1980}. Methodologies like Gaia provide organizational abstractions (roles, interactions, norms) useful for analysis and design \parencite{Zambonelli2003}.

For heterogeneous enterprise landscapes (ERP, data warehouses, SaaS, legacy), multi-agent systems offer modularity and fault isolation: specialized agents encapsulate capabilities (e.g., document understanding, policy checking, posting to ERP), coordinate via protocols, and scale horizontally. However, integration and governance remain hard problems: ensuring interoperability with legacy systems, maintaining compliance and audit trails, and making agent decisions transparent enough for risk and regulatory stakeholders \parencite{Luck2005}. The recent infusion of GenAI amplifies multi-agent systems potential—agents can reason with unstructured artifacts, learn task patterns, and collaborate—but simultaneously raises the bar for safety, observability, and evaluation.
\section{Applicability Conditions for Agentic Automation} 
\subsection{Workflow Sustainability Criteria}
\subsection{Use Case Illustration: Finance Domain}
\subsection{Use Case Validation of the Evaluation Framework}
\newpage  
\section{Conceptual Architecture Design}\label{sec:5architectDesign}
\subsection{Agents Roles and Capabilities}
\subsection{Agents Roles, Behaviors, \& System Embedding}
\subsection{Multi-Agent Interaction \& Orchestration Model}\label{sec:conceptualarchitecture}
\newpage 
\section{Conclusion}