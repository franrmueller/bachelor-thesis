\section{Introduction}
Organizations across industries continue to face challenges in achieving operational excellence due to fragmented processes, manual interventions, and inconsistent data quality. Legacy workflows and siloed systems reinforce these inefficiencies, while traditional automation approaches often lack the adaptability required in dynamic business environments. Generative artificial intelligence (GenAI) offers a new opportunity to not only automate but also extend and redesign processes, improving decision-making and governance. However, many organizations lack structured strategies for embedding these capabilities into their operational workflows in a scalable and value-driven manner. This thesis addresses this gap by designing a multi-agent architecture that integrates GenAI into workflow automation, with the aim of enhancing agility, compliance, and decision quality in pursuit of operational excellence.

The latest iteration of artificial intelligence is generative agents, systems that integrate language models with external tools, enabling autonomous task execution with increasing complexity and flexibility. This thesis explores how and when to deploy such agents to drive meaningful improvements in process quality, agility, and governance. It investigates the following central research question:

\vspace{0.5\baselineskip}
\emph{How can a multi-agent architecture be designed to integrate GenAI capabilities into workflow automation, in order to enhance agility, compliance, \& decision quality to achieve operational excellence?}
\vspace{0.5\baselineskip}

The following sub-questions further specify the scope of investigation:
\begin{itemize}
    \item \emph{What are the strengths \& limitations of GenAI in the context of workflow automation using a multi-agent architecture?}
    \item \emph{Which design requirements \& agent roles are necessary to align a multi-agent architecture with the goals of operational excellence?}
    \item \emph{Under which conditions is deploying a generative multi-agent architecture justified over traditional automation approaches?}
\end{itemize}

The core contribution of this work is the conceptual design of a reference architecture for a multi-agent system (MAS) that leverages GenAI to support operational excellence in enterprise workflow automation. This includes:
\begin{enumerate}
    \item A structured synthesis of system requirements derived from academic literature and operational excellence principles.
    \item A conceptual architecture detailing agent roles, interactions, and integration points
    \item A set of applicability conditions and design considerations to guide future deployment and evaluation of generative multi-agent architectures in practice.
\end{enumerate}

Both the formal evaluation of the architecture and the development of the system are out of scope due to time constraints, but are proposed as directions for future work.


The approach is industry-agnostic and client-centric, drawing illustrative examples from the financial services sector; a domain characterized by complex processes, a high regulatory burden, and significant challenges for legacy systems.

This document begins by introducing the research problem and then progressively guides the reader through its theoretical and practical foundations. Section 2 outlines the research methodology, including design science research (DSR) and supporting sub-methods. Section 3 presents a review of relevant literature. Section 4 focuses on the design of the proposed MAS architecture, and Section 5 explores applicability conditions and design implications. The thesis concludes with a reflection on the broader implications of the findings and offers suggestions for future research and application.


\section{Methodology} \label{2methodology}
This research adopts the DSR methodology, which is particularly suitable for investigating the application of information systems (IS) in business contexts. In support of the different phases of the study, qualitative content analysis, requirements engineering (RE), and process modeling are additionally employed.

The literature review follows the qualitative content analysis approach proposed by \textcite{mayringQualitativeContentAnalysis2022} to extract and synthesize relevant knowledge. However, since this method is considered standard practice in a thesis, it has no dedicated subsection.

\subsection{Design Science Research Methodology} \label{subsec:dsrm}
This thesis applies DSR methodology to develop a conceptual architecture for a generative MAS aimed at enterprise workflow automation. DSR is well-suited for producing prescriptive artifacts—constructs, models, methods, and instantiations—that solve defined organizational problems \parencite{hevnerDesignScienceInformation2004}.

In this work:

\begin{itemize}
    \item \emph{Constructs} represent the MAS-related concepts identified (e.g., Orchestration Agent, Compliance Verification Agent, Data Integration Agent).
    \item The \emph{Model} is the conceptual architecture integrating these constructs into a coherent system design (Section \ref{sec:5architectDesign}).
    \item The \emph{Method} (algorithm or guideline) gives life to the model by explaining the steps toward performing the desired task. \textcite{marchDesignNaturalScience1995} use the example of data structures, which combine different data types with algorithms to store and retrieve data.
    \item An \emph{Instantiation} is the actual realization of an artifact in the real world. It requires the translation of constructs, models, and methods into working software or systems that address the business problem.
\end{itemize}

In contrast to natural science, design science is more concerned with utility than truth. Researchers build an artifact to demonstrate feasibility, either by proving that a viable solution exists for an unaddressed problem or by improving upon existing solutions. This artifact then becomes the object of study. In many business contexts, artifacts are created out of necessity without a formal methodological process. DSR can be used retrospectively to reconstruct the underlying constructs, models, and methods in order to evaluate and share their utility \parencite[cf.][pp. 256–259]{marchDesignNaturalScience1995}.

This work follows the DSR process defined by \textcite{peffersDesignScienceMethodology2007}, consisting of the following activities:

\begin{enumerate}
    \item \emph{Problem identification}: Identifying a problem or set of problems that need to be addressed. Proper fragmentation of the problem into concrete issues helps define the project scope.
    \item \emph{Objective identification}: Inferring objectives from the problem definition and literature. These are later transformed into criteria for evaluating the success of the artifact. Objectives can be qualitative (e.g., introducing a new paradigm) or quantitative (e.g., meeting a KPI threshold).
    \item \emph{Design \& development}: Designing and creating the artifact, which may consist of constructs, models, and methods. This thesis focuses on conceptual design rather than instantiation.
    \item \emph{Demonstration}: Applying the artifact in a practical scenario to illustrate its utility. This can take place in a real-world setting or controlled experiment.
    \item \emph{Evaluation}: Assessing the artifact using criteria defined in phase two. Evaluation tools may include test scenarios or simulations. Iterations between phases two to five are common until the artifact reaches satisfactory maturity.
    \item \emph{Communication}: Communicating the artifact and findings with scientific rigor to both technical experts and decision-makers.
\end{enumerate}

Although these steps represent the nominal sequence, the DSR process is not strictly linear. \textcite[2007][pp. 52--56]{peffersDesignScienceMethodology2007} describe four entry points into the process: problem-centered, objective-centered, design- and development-centered, and client/context-initiated. This thesis follows a \emph{design-centered approach}, meaning the research is guided by the capabilities and constraints of existing but emerging technologies. The goal is to derive design requirements from relevant literature and operational excellence principles, and to conceptually design a multi-agent architecture that leverages these technologies. Illustrative use cases---drawn from the financial services sector---are used to ground the discussion, while maintaining the architecture’s \emph{industry-agnostic} and \emph{client-centric} scope. The scope is limited to the conceptual design of a method and architecture for a MAS; development, demonstration, and evaluation are proposed as future work. To support this design, the next section outlines how relevant system requirements were identified and structured.


DSR was selected over alternative research strategies such as case study or action research because it is well-suited to the creation of prescriptive artifacts aimed at solving defined organizational problems. This aligns with the thesis objective of producing a conceptual architecture rather than solely observing or intervening in an existing setting.

\subsection{Requirements Engineering \& System Analysis}
\label{subsec:requirements}
RE represents the initial practical phase in designing an information technology system. It translates the abstract definition of a problem into concrete specifications—referred to as requirements—that describe what an IT system should do before it exists. A requirement is a condition or feature that the system must meet to assist the user in performing a task or achieving a goal. A contract, service-level agreement, or government regulation is also a form of requirement \parencite[cf.][p. 62]{IEEEStandardGlossary1991}.

In the context of this thesis, requirements are formulated for a conceptual MAS intended to support enterprise workflow automation. These requirements can be classified into three main categories. \emph{Functional requirements} describe the essential characteristics needed for the system to achieve its intended purpose. \emph{Quality requirements} improve system performance or usability (e.g., resource optimization, user experience, or processing speed). \emph{Constraints} define external boundaries such as budget, time, dependencies, and system integration \parencite[cf.][p. 8]{glinzRequirementEngineering2020}.

In this thesis, requirements are synthesized from academic literature, technical documentation, and case studies. This aligns with the artifact-centered nature of DSR, where requirements emerge from the problem context and guide the conceptual design. The RE process follows three main stages:

\begin{enumerate}
    \item \emph{Contextual Framing}: Analyzing the problem space by identifying recurring challenges, architectural needs, and strategic goals related to operational excellence and workflow automation in enterprise environments.
    \item \emph{Requirements Extraction}: Extracting relevant requirements through qualitative literature review and document analysis. Sources include publications on agentic systems, enterprise architecture frameworks, ERP documentation, and industry use cases.
    \item \emph{Requirements Structuring \& Traceability}: Clustering and prioritizing requirements and linking them to the system architecture. This ensures design decisions are grounded in the problem definition and overall research objectives.
\end{enumerate}

The outcome of the RE process directly informs the system design, laying the foundation for the conceptual MAS architecture presented in the next subsection. The resulting requirements are explicitly linked back to the central research question and sub-questions, ensuring traceability from problem identification through to the proposed architecture design. This ensures ongoing aligment between the research objectives and the artifact’s conceptual structure.

\subsection{Information System Design}
Within the DSR process, this stage corresponds to the \emph{design \& development} activity as defined by \textcite{peffersDesignScienceMethodology2007}. While Section~\ref{subsec:dsrm} outlined the overall research cycle, the focus here is on the methodological approach used to transform the structured requirements from Section~\ref{subsec:requirements} into a formal conceptual artifact. The outcome of this stage is a \emph{model} in the DSR sense \parencite{peffersDesignScienceMethodology2007}, representing a prescriptive solution that can be evaluated, communicated, and eventually instantiated.

From a methodological perspective, information system design (ISD) is not an ad-hoc creative process but a structured, model-driven activity. \textcite[cf.][p. 68]{hevnerDesignScienceInformation2004} define ISD as both a process (comprising expert activities that build and evaluate artifacts) and a product (represented by constructs, models, methods, or instantiations aimed at solving an identified problem). ISD translates an abstract, requirement-oriented view of a system into a structured specification that can be implemented. Situated between requirements analysis and implementation, ISD ensures that all functional, quality, and constraint requirements are accurately reflected in the proposed solution concept.

By explicitly linking each design element back to its originating requirement, the approach supports both \emph{validation} (ensuring the right system is built) and \emph{verification} (ensuring the system is built right), while maintaining adaptability to changing operational conditions. In a research context, the design step produces conceptual models that form the basis for communication among stakeholders, evaluation against requirements, and eventual realization \parencite{wand1995ontology}.

In this thesis, the conceptual design process follows \emph{model-based systems engineering} (MBSE) principles, employing the Systems Modeling Language (SysML) as the primary modeling notation. SysML, an Object Management Group standard derived from UML and adapted for systems engineering \parencite{omgsysml}, supports both \emph{structural} views (e.g., block definition diagrams, internal block diagrams) and \emph{behavioral} views (e.g., activity, sequence, and state machine diagrams). It also provides dedicated requirement diagrams that enable explicit traceability between requirements and design elements through mechanisms such as the \texttt{\guillemotleft satisfy\guillemotright} and \texttt{\guillemotleft verify\guillemotright} relationships \parencite{friedenthal2014practical}. This traceability ensures that, for example, a ``Compliance Verification Agent'' modeled in a block definition diagram can be directly linked to quality and constraint requirements concerning auditability and regulatory compliance.

SysML was selected over alternative notations such as UML, ArchiMate, or BPMN for three primary reasons: (1) it offers an integrated approach to structural, behavioral, and requirement modeling within a single framework; (2) it is vendor-neutral and widely adopted in both academia and industry, ensuring long-term relevance; and (3) it is well-suited to multi-domain systems that combine software agents, business processes, and IT infrastructure—critical for the architecture envisioned in this research. While UML excels in software-level specification, it lacks native requirement modeling; ArchiMate is optimized for enterprise architecture abstraction but offers less granularity in behavioral modeling; and BPMN focuses on process flows without structural integration. The integrated nature of SysML is especially advantageous for operational excellence initiatives, where governance, compliance, and continuous improvement must be considered alongside technical design.

In the context of this thesis, the application of SysML is scoped to the conceptual level. The modeling activities include:
\begin{itemize}
    \item \emph{Requirement Diagrams} to link each modeled component to specific functional, quality, and constraint requirements from Section~\ref{sec:requirements}.
    \item \emph{Block Definition Diagrams} to represent the high-level system structure and main agent types (e.g., Orchestration Agent, Compliance Verification Agent, Data Integration Agent).
    \item \emph{Activity Diagrams} to illustrate core interaction patterns between agents and external systems, ensuring alignment with orchestration and governance principles.
\end{itemize}

By adopting a formalized, traceability-driven modeling approach, the ISD stage ensures that the proposed artifact is rigorously specified, grounded in the problem definition, and evaluable against explicit criteria. The resulting architecture views are presented in Section~\ref{sec:conceptualarchitecture}.


\section{Literature Review}
\label{sec:literature-review}

This section synthesizes academic and industry perspectives on three pillars underpinning this research: (i) the evolving definition of operational excellence, (ii) the shift in enterprise automation from rules-based RPA to goal-oriented generative agents, and (iii) the role of multi-agent systems in complex enterprise settings. The review follows a qualitative content analysis approach \parencite{mayringQualitativeContentAnalysis2022} and remains problem-oriented: each subsection surfaces implications for the requirements engineered in Section~2.2 and the architecture designed in Section~5.

\subsection{The Evolving Definition of Operational Excellence}
Operational excellence (OpEx) emerged from manufacturing disciplines—Lean and Six Sigma—where the primary levers were waste elimination, variance reduction, and standard work \parencite{Womack1990,Harry1998}. Contemporary excellence models broaden this focus to services and knowledge work, emphasizing culture, strategy alignment, governance, and learning. Representative references include the Shingo Model and the EFQM Model, both of which integrate people, process, and result dimensions while stressing continuous improvement and adaptability \parencite{Shingo2014,EFQM2020}.

In data-rich and highly regulated domains (e.g., finance), the OpEx constraint is increasingly cognitive rather than purely procedural: human attention, handoffs, and decision latency limit throughput and quality \parencite{Hammer2004,Davenport2018}. Yet most excellence frameworks still presuppose human-led improvement cycles, offering limited guidance on embedding autonomous, machine-driven capabilities for sustained optimization, monitoring, and compliance. This gap motivates evaluating whether agentic, tool-using AI can extend OpEx from periodic human interventions toward continuous, machine-assisted improvement and control. For this thesis, the implications are twofold: requirements should (1) include decision-quality, compliance, and auditability as first-class non-functional goals, and (2) anticipate organizational enablers (roles, governance) that allow AI systems to participate in improvement cycles.

\subsection{Automation Paradigms: From RPA to GenAI Agents}
Robotic Process Automation (RPA) has proven effective for stable, rule-based back-office tasks and has delivered measurable cost, speed, and quality benefits \parencite{Lacity2016}. However, RPA’s brittleness under interface or policy changes and its limited handling of unstructured inputs constrain scalability in dynamic environments \parencite{Syed2021}. ``Intelligent'' variants (often termed IPA) augment RPA with OCR, NLP, and ML components, broadening applicability but remaining predominantly workflow- and rule-driven.

GenAI agents mark a paradigm shift: powered by large language models and tool orchestration, agents can decompose goals into sub-tasks, call external tools/APIs, reflect on intermediate outputs, and adapt strategies during execution \parencite{Park2023,Rodriguez_Agents_2025}. Conceptually, this elevates automation from \emph{process mimicry} (predefined steps) to \emph{goal-oriented problem solving} (contextual reasoning under uncertainty). The literature highlights open challenges central to enterprise adoption—reliability under distribution shift, controllability, explainability, data protection, and governance of autonomous actions \parencite{Bommasani2022}. For system design, this motivates requirements for guardrails (policy constraints, approvals), observability (telemetry, traces), and life-cycle management (versioning, evaluation) that exceed what is typical for RPA.

\subsection{Multi-Agent Systems in Enterprise Contexts}
MAS comprise autonomous, interacting agents that perceive, decide, and act within a shared environment \parencite{Wooldridge2009}. MAS have long been proposed for enterprise-scale coordination problems—supply chains, scheduling, and distributed control—leveraging decomposition, local decision-making, and coordination protocols \parencite{Parunak1999,JenningsBussmann2003}. Established patterns include the blackboard architecture for shared problem solving \parencite{Nii1986} and market/negotiation mechanisms such as the Contract Net Protocol for dynamic task allocation \parencite{Smith1980}. Methodologies like Gaia provide organizational abstractions (roles, interactions, norms) useful for analysis and design \parencite{Zambonelli2003}.

For heterogeneous enterprise landscapes (ERP, data warehouses, SaaS, legacy), MAS offer modularity and fault isolation: specialized agents encapsulate capabilities (e.g., document understanding, policy checking, posting to ERP), coordinate via protocols, and scale horizontally. However, integration and governance remain hard problems: ensuring interoperability with legacy systems, maintaining compliance and audit trails, and making agent decisions transparent enough for risk and regulatory stakeholders \parencite{Luck2005}. The recent infusion of GenAI amplifies MAS potential—agents can reason with unstructured artifacts, learn task patterns, and collaborate—but simultaneously raises the bar for safety, observability, and evaluation.

\section{Applicability Conditions for Agentic Automation} 
\subsection{Workflow Sustainability Criteria}
\subsection{Use Case Illustration: Finance Domain}
\subsection{Use Case Validation of the Evaluation Framework}
\newpage  
\section{Conceptual Architecture Design} \label{sec:5architectDesign}
\subsection{Agents Roles and Capabilities}
\subsection{Agents Roles, Behaviors, \& System Embedding}
\subsection{Multi-Agent Interaction \& Orchestration Model}
\label{sec:conceptualarchitecture}
\newpage 
\section{Conclusion}