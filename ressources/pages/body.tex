\section{Introduction}\label{sec:intro}
Organizations across industries continue to face persistent challenges in achieving operational excellence. Fragmented processes, manual interventions, and inconsistent data quality undermine efficiency and decision-making. Legacy workflows and siloed systems exacerbate these inefficiencies, while traditional automation approaches often lack the adaptability needed in dynamic business environments. For companies, this translates into slower response times, higher compliance risks, and limited scalability—issues that directly threaten competitiveness.

Generative artificial intelligence (GenAI) opens new possibilities to extend automation beyond deterministic scripts, enabling adaptive, tool-using agents that support governance, decision quality, and agility. However, despite this potential, organizations and the academic literature alike lack structured strategies and conceptual frameworks for embedding such agentic capabilities into operational workflows in a scalable, value-driven way. This gap motivates the present research.

This thesis investigates how multi-agent systems can serve as a reference architecture for integrating GenAI into enterprise workflow automation. The central research question is:

\vspace{0.5\baselineskip}
\emph{How can a multi-agent architecture be designed to integrate GenAI capabilities into workflow automation, in order to enhance agility, compliance, \& decision quality to achieve operational excellence?}
\vspace{0.5\baselineskip}

To answer this question, the study addresses the following sub-questions:
\begin{itemize}
    \item \emph{What are the strengths \& limitations of GenAI in the context of workflow automation using a multi-agent architecture?}
    \item \emph{Which design requirements \& agent roles are necessary to align a multi-agent architecture with the goals of operational excellence?}
    \item \emph{Under which conditions is deploying a generative multi-agent architecture justified over traditional automation approaches?}
\end{itemize}

Methodologically, the thesis applies Design Science Research (DSR) to develop a conceptual reference architecture. The approach synthesizes requirements from academic literature and operational excellence principles, models agent roles and interactions, and derives applicability conditions for real-world deployment.

The core contribution of this work is a conceptual design of a multi-agent system that leverages GenAI to support operational excellence in enterprise workflows. Specifically, it delivers:
\begin{enumerate}
    \item A structured synthesis of system requirements derived from academic literature and operational excellence principles.
    \item A conceptual architecture detailing agent roles, interactions, and integration points
    \item A set of applicability conditions and design considerations to guide future deployment and evaluation of generative multi-agent architectures in practice.
\end{enumerate}

The scope is limited to conceptual design; formal evaluation and technical implementation are proposed as future work. The approach remains industry-agnostic but draws illustrative examples from the financial services sector, given its regulatory complexity and reliance on legacy systems.

The thesis is structured as follows: Section 2 outlines the research methodology, including DSR and supporting methods. Section 3 presents a literature review on operational excellence, automation paradigms, and multi-agent systems. Section 4 develops applicability conditions and use case illustrations. Section 5 introduces the conceptual architecture design, and Section 6 concludes with reflections and directions for future research.

\section{Methodology}\label{sec:method}
This thesis applies the design science research methodology to create a conceptual artifact: a multi-agent architecture for workflow automation. Practically, the approach unfolded in three steps: (1) reviewing the literature on operational excellence, workflow automation, and agentic AI; (2) deriving and structuring requirements from literature and case material into a requirements model; and (3) designing a conceptual system architecture using System Modeling Language (SysML).

Supporting methods included Mayring-style qualitative content analysis (QCA) for the review, requirements engineering (RE) and systems analysis for the requirements model, and MBSE to structure the architecture and ensure requirement-to-design traceability. Within DSR, the work deliberately focuses on problem identification, objective definition, and conceptual design, while instantiation/demonstration and formal evaluation are out of scope given the bachelor-thesis format and resource constraints. This scoping maintains methodological rigor while keeping the contribution focused: a well-argued reference architecture ready for subsequent implementation and empirical evaluation.

\subsection{Qualitative Content Analysis}\label{subsec:qca}
To ensure a structured literature review, this thesis employed qualitative content analysis following \textcite{mayringQualitative2022}. 
QCA offers a transparent, rule-based procedure for synthesizing knowledge from textual sources while retaining interpretative depth. 
In this work it supports the DSR process \parencite{peffersDesignScienceMethodology2007} within the \emph{problem identification and motivation} phase, 
where the aim is to understand the state of the problem domain and justify the value of a solution.

\paragraph*{Analytical Units}
Following Mayring, the analytical framework was defined prior to coding:
\begin{itemize}
    \item \textit{Analysis unit}: the overall literature corpus addressing operational excellence, workflow automation, and agentic AI.
    \item \textit{Context unit}: individual publications (books, peer-reviewed articles, standards, industry reports, conference transcripts, and case studies).
    \item \textit{Coding unit}: discrete statements or conceptual claims relevant to the intersection of operational excellence, automation paradigms, and AI-based multi-agent systems.
\end{itemize}

\paragraph*{Category Development}
A \emph{mixed deductive--inductive} approach was used. Deductive categories were derived from established theory, including operational excellence dimensions (adaptability, compliance, decision quality) and prior automation frameworks (RPA, IPA). 
Inductive categories were generated from the material itself, capturing emerging issues such as ``guardrails,'' ``observability,'' and ``traceability'' in agentic AI systems. 
Coding followed Mayring's rule-governed categorization to ensure consistency and avoid arbitrary interpretation.

\paragraph*{Integration into the DSR Process}
The resulting categories served two functions:
\begin{enumerate}
    \item \textit{Problem representation}: categories structured how the research problem was represented, aligning with \textcite{hevnerDesignScienceInformation2004}, who emphasize that effective constructs are essential to problem framing.
    \item \textit{Derivation of objectives}: categories were transformed into metarequirements that guided the definition of solution objectives in Activity~2 of the DSR methodology \parencite{peffersDesignScienceMethodology2007}.
\end{enumerate}

\paragraph*{Practical Application}
The review was conducted by systematically coding the literature across the three pillars (OpEx, automation paradigms, MAS). 
For example, claims such as ``RPA is brittle under interface changes'' were coded under the deductive category \emph{limitations of RPA}, 
while repeated references to audit trails and logging practices were inductively grouped under \emph{traceability}. 
For each publication, relevant statements were assigned to categories using predefined coding rules. 
The resulting category set both structures Chapter~\ref{sec:lit-rev} and forms the basis for the requirements engineering in Section~\ref{subsec:re-sa}.

\subsection{Requirements Engineering \& System Analysis}\label{subsec:re-sa}

\subsection{Model-Based Systems Engineering}\label{subsec:mbse}

\section{Literature Review}\label{sec:lit-rev}

\subsection{Operational Excellence}\label{subsec:op-ex}
Operational excellence is a multifaceted concept centered on continuous improvement, adptability, and delivering value. Mayring's QCA approach allowed the definition of clear analytical units (e.g. definitions and principles of OpEx from each source), apply deductive codes from established theory (e.g. lean principles, quality management frameworks), and derive inductive codes emerging specifically in IT settings and automation contexts. The main categories extracted include \emph{agility}, \emph{compliance}, and \emph{decision quality}, which support and inform the design of requirements in this design science research project.

Operational excellence is a holistic management philosophy emerged in the late 20th century, aiming to improve quality and efficiency in manufacturing, specially in the automovilistic industry. It emphasizes continuous improvement, customer value, and adaptability \parencite{womackLean2013}. It integrates principles from lean manufacturing, Six Sigma, and Total Quality Management (TQM) to optimize processes, reduce waste, and enhance quality \parencite{juranQualityControlHandbook2010}. In recent years, the concept has expanded to include agility and digital transformation as key components \parencite{hammerWhatIsBusiness2007}.
\paragraph{Agility \& Adaptability}

\paragraph{Compliance & Risk Management}
\paragraph{Decision Quality}
\paragraph{Efficiency and Continuous Improvement}
\paragraph{Customer-Centricity}
\paragraph{User Empowerment \& Collaboration}
\paragraph{Technology Integration \& Scalability}