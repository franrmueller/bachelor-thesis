% Bibliography
\clearpage
\phantomsection\
\addcontentsline{toc}{section}{References}
\markboth{References}{References}
\section*{References}
\printbibliography[heading=none]

% --- Appendix ---
\clearpage
\appendix

\phantomsection
\addcontentsline{toc}{section}{Appendix}

\refstepcounter{section}\label{sec:appendix}
\setcounter{subsection}{0}
\markboth{Appendix}{Appendix}
\section*{Appendix}

\addtocontents{toc}{\protect\setcounter{tocdepth}{1}} 

\subsection*{Requirements List}\label{app:req-list}
\subsubsection*{Organized by Requirement Type}
\begin{footnotesize}
  \textsc{functional requirements}
    \begin{itemize}
        \item \textsc{FR-01 process orchestration engine} --- The system shall execute workflow models by dispatching tasks to human or software actors according to model control flow and business rules.
        \item \textsc{FR-02 parameterized subprocesses} --- The system shall support parameterized subprocesses and rules so behavioral variants can be expressed without altering the underlying process structure.
        \item \textsc{FR-03 human-machine task orchestration} --- The system shall coordinate both human and automated tasks within the same orchestrated process model.
        \item \textsc{FR-04 exception detection and routing} --- The system shall detect deviations from the nominal process and route cases to defined exception subprocesses.
        \item \textsc{FR-05 escalation to human authority} --- The system shall escalate unresolved or unmodeled exceptions to designated human roles with clear ownership.
        \item \textsc{FR-06 inter-agent communication protocol} --- The system shall specify message formats and interaction rules for agent collaboration (e.g., direct messaging or shared memory; event-bus/blackboard where asynchronous exchange is required).
        \item \textsc{FR-07 tool failure handling in exceptions} --- The system shall validate tool/API outputs and trigger defined recovery or fallback steps when invocations fail.
        \item \textsc{FR-08 conflict resolution} --- The system shall provide mechanisms for agents to resolve task or decision conflicts (e.g., delegation to a planner, negotiation, or voting).
        \item \textsc{FR-09 policy engine} --- The system shall provide a decoupled policy evaluation component that can validate, veto, or redirect workflow executions and agent actions at runtime.
        \item \textsc{FR-10 risk-based approvals and escalation} --- The system shall require human approval and/or escalation for actions or decisions exceeding defined risk or impact thresholds.
        \item \textsc{FR-11 compliance mapping and drift checks} --- The system shall map actions and policies to applicable regulations or internal rules and perform runtime checks to detect and block compliance drift.
        \item \textsc{FR-12 evidence-based decision support} --- The system shall aggregate relevant data at decision points to support evidence-based choices and reduce errors.
        \item \textsc{FR-13 rule-enforced decision points} --- The system shall enforce business rules and role responsibilities at decision points to ensure consistent, auditable outcomes.
        \item \textsc{FR-14 decision trace and rationale} --- The system shall persist a human-readable decision trace for automated or assisted decisions, including inputs, tool/API calls and results, and concise rationale summaries.
        \item \textsc{FR-15 event logging pipeline} --- The system shall instrument the workflow engine and agents to emit structured, timestamped events for key actions (e.g., prompts, tool/API invocations and results, plan/decision commits, and state changes).
        \item \textsc{FR-16 dashboards and alerts} --- The system shall provide real-time dashboards and alerting for observability metrics (e.g., execution latency, error/anomaly rates, blocked actions).
        \item \textsc{FR-17 replay for post-hoc analysis} --- The system shall support reconstruction and replay of workflow and agent interactions from logged events to enable root-cause analysis and explanation of outcomes.
        \item \textsc{FR-18 integration connectors} --- The system shall provide connectors to integrate heterogeneous applications and data sources required by the workflows.
        \item \textsc{FR-19 agent tool adapters} --- The system shall expose a uniform adapter interface for agents and workflows to invoke external tools, APIs, databases, or RPA scripts.
        \item \textsc{FR-20 inter-organizational interoperability} --- The system shall support protocol and interface interoperability suitable for cross-organizational workflows.
        \item \textsc{FR-21 data transformation layer} --- The system shall provide mapping and transformation capabilities to reconcile data across integrated systems.
    \end{itemize}

    \noindent \textsc{constraints}
    \begin{itemize}
        \item \textsc{C-01 segregation of duties} --- The system shall enforce role-based access control and segregation of duties for configuration changes and sensitive actions.
        \item \textsc{C-02 audit logging and retention} --- The system shall produce tamper-evident audit trails of agent actions, policy decisions, and configuration changes, retained according to the applicable compliance policy.
        \item \textsc{C-03 bounded decision autonomy} --- The system shall constrain agent decision-making to clearly scoped authority levels aligned with organizational objectives.
        \item \textsc{C-04 modular process units} --- The system shall structure workflows as modular subprocesses or services that can be reused and reconfigured without redesign.
        \item \textsc{C-05 interface-based composition} --- The system shall expose clear process and service interfaces compatible with established interoperability standards to enable composition across heterogeneous systems and organizations.
        \item \textsc{C-06 explicit coordination model} --- The system shall define and enforce a coordination structure (e.g., hierarchical planner-specialists or decentralized collaboration) for multi-agent work.
        \item \textsc{C-07 agent status self-reporting} --- The system shall require agents to periodically self-report status and progress (e.g., current task, step outcome, next planned action) to improve runtime transparency.
        \item \textsc{C-08 interface contracts and schemas} --- The system shall define input/output contracts and validate request/response schemas at adapter boundaries.
        \item \textsc{C-09 invocation safeguards} --- The system shall enforce adapter-level safeguards (e.g., timeouts, retries, and idempotency keys) to limit side effects of failed or repeated tool calls.
    \end{itemize}
\end{footnotesize}

\noindent Organized by Requirement Cluster \\ \\
\begin{footnotesize}
  \textsc{governance and compliance}
    \begin{itemize}
      \item \textsc{FR-09 policy engine} --- The system shall provide a decoupled policy evaluation component that can validate, veto, or redirect workflow executions and agent actions at runtime.
      \item \textsc{FR-10 risk-based approvals and escalation} --- The system shall require human approval and/or escalation for actions or decisions exceeding defined risk or impact thresholds.
      \item \textsc{FR-11 compliance mapping and drift checks} --- The system shall map actions and policies to applicable regulations or internal rules and perform runtime checks to detect and block compliance drift.
      \item \textsc{C-01 segregation of duties} --- The system shall enforce role-based access control and segregation of duties for configuration changes and sensitive actions.
      \item \textsc{C-02 audit logging and retention} --- The system shall produce tamper-evident audit trails of agent actions, policy decisions, and configuration changes, retained according to the applicable compliance policy.
    \end{itemize}
  \textsc{decision quality}
    \begin{itemize}
      \item \textsc{FR-12 evidence-based decision support} --- The system shall aggregate relevant data at decision points to support evidence-based choices and reduce errors.
      \item \textsc{FR-13 rule-enforced decision points} --- The system shall enforce business rules and role responsibilities at decision points to ensure consistent, auditable outcomes.
      \item \textsc{C-03 bounded decision autonomy} --- The system shall constrain agent decision-making to clearly scoped authority levels aligned with organizational objectives.
    \end{itemize}
  \textsc{orchestration and modularity}
    \begin{itemize}
      \item \textsc{FR-01 process orchestration engine} --- The system shall execute workflow models by dispatching tasks to human or software actors according to model control flow and business rules.
      \item \textsc{FR-02 parameterized subprocesses} --- The system shall support parameterized subprocesses and rules so behavioral variants can be expressed without altering the underlying process structure.
      \item \textsc{FR-03 human-machine task orchestration} --- The system shall coordinate both human and automated tasks within the same orchestrated process model.
      \item \textsc{C-04 modular process units} --- The system shall structure workflows as modular subprocesses or services that can be reused and reconfigured without redesign.
      \item \textsc{C-05 interface-based composition} --- The system shall expose clear process and service interfaces compatible with established interoperability standards to enable composition across heterogeneous systems and organizations.
    \end{itemize}
  \textsc{exception handling and coordination}
    \begin{itemize}
      \item \textsc{FR-04 exception detection and routing} --- The system shall detect deviations from the nominal process and route cases to defined exception subprocesses.
      \item \textsc{FR-05 escalation to human authority} --- The system shall escalate unresolved or unmodeled exceptions to designated human roles with clear ownership.
      \item \textsc{FR-06 inter-agent communication protocol} --- The system shall specify message formats and interaction rules for agent collaboration (e.g., direct messaging or shared memory; event-bus/blackboard where asynchronous exchange is required).
      \item \textsc{FR-07 tool failure handling in exceptions} --- The system shall validate tool/API outputs and trigger defined recovery or fallback steps when invocations fail.
      \item \textsc{FR-08 conflict resolution} --- The system shall provide mechanisms for agents to resolve task or decision conflicts (e.g., delegation to a planner, negotiation, or voting).
      \item \textsc{C-06 explicit coordination model} --- The system shall define and enforce a coordination structure (e.g., hierarchical planner-specialists or decentralized collaboration) for multi-agent work.
    \end{itemize}
    \textsc{observability and traceability}
      \begin{itemize}
        \item \textsc{FR-14 decision trace and rationale} --- The system shall persist a human-readable decision trace for automated or assisted decisions, including inputs, tool/API calls and results, and concise rationale summaries.
        \item \textsc{FR-15 event logging pipeline} --- The system shall instrument the workflow engine and agents to emit structured, timestamped events for key actions (e.g., prompts, tool/API invocations and results, plan/decision commits, and state changes).
        \item \textsc{FR-16 dashboards and alerts} --- The system shall provide real-time dashboards and alerting for observability metrics (e.g., execution latency, error/anomaly rates, blocked actions).
        \item \textsc{FR-17 replay for post-hoc analysis} --- The system shall support reconstruction and replay of workflow and agent interactions from logged events to enable root-cause analysis and explanation of outcomes.
        \item \textsc{C-07 agent status self-reporting} --- The system shall require agents to periodically self-report status and progress (e.g., current task, step outcome, next planned action) to improve runtime transparency.
      \end{itemize}
    \textsc{tool integration}
      \begin{itemize}
        \item \textsc{FR-18 integration connectors} --- The system shall provide connectors to integrate heterogeneous applications and data sources required by the workflows.
        \item \textsc{FR-19 agent tool adapters} --- The system shall expose a uniform adapter interface for agents and workflows to invoke external tools, APIs, databases, or RPA scripts.
        \item \textsc{FR-20 inter-organizational interoperability} --- The system shall support protocol and interface interoperability suitable for cross-organizational workflows.
        \item \textsc{FR-21 data transformation layer} --- The system shall provide mapping and transformation capabilities to reconcile data across integrated systems.
        \item \textsc{C-08 interface contracts and schemas} --- The system shall define input/output contracts and validate request/response schemas at adapter boundaries.
        \item \textsc{C-09 invocation safeguards} --- The system shall enforce adapter-level safeguards (e.g., timeouts, retries, and idempotency keys) to limit side effects of failed or repeated tool calls.
    \end{itemize}
\end{footnotesize}

\clearpage
\subsection*{Requirements SysML v2 Model}\label{app:req-mod}
\section*{Note on Modeling Artifacts}
The SysML v2 listings were generated by the author using Eclipse-based tooling. 
Large language model assistants were used during the iterative debugging of syntax, 
but the final model was reviewed, corrected, and validated manually. 
This is documented in the KI-Verzeichnis.

\begin{figure}[htbp]
  \centering
  \includegraphics[width=\linewidth]{ressources/MAS/figures/MASRequirements/MASRequirements1.jpeg}
  \caption{SysML v2 requirements package overview (1/6).}
  \label{fig:mas-reqs-1}
\end{figure}
\begin{figure}[htbp]
  \centering
  \includegraphics[width=\linewidth]{ressources/MAS/figures/MASRequirements/MASRequirements2.jpeg}
  \caption{SysML v2 requirements package overview (2/6).}
  \label{fig:mas-reqs-2}
\end{figure}
\begin{figure}[htbp]
  \centering
  \includegraphics[width=\linewidth]{ressources/MAS/figures/MASRequirements/MASRequirements3.jpeg}
  \caption{SysML v2 requirements package overview (3/6).}
  \label{fig:mas-reqs-3}
\end{figure}
\begin{figure}[htbp]
  \centering
  \includegraphics[width=\linewidth]{ressources/MAS/figures/MASRequirements/MASRequirements4.jpeg}
  \caption{SysML v2 requirements package overview (4/6).}
  \label{fig:mas-reqs-4}
\end{figure}
\begin{figure}[htbp]
  \centering
  \includegraphics[width=\linewidth]{ressources/MAS/figures/MASRequirements/MASRequirements5.jpeg}
  \caption{SysML v2 requirements package overview (5/6).}
  \label{fig:mas-reqs-5}
\end{figure}
\begin{figure}[htbp]
  \centering
  \includegraphics[width=\linewidth]{ressources/MAS/figures/MASRequirements/MASRequirements6.jpeg}
  \caption{SysML v2 requirements package overview (6/6).}
  \label{fig:mas-reqs-6}
\end{figure}

\clearpage
\lstinputlisting[
  style=sysml,
  label={lst:mas-reqs-src},
  numbers=left,
  inputencoding=utf8
]{./ressources/MAS/MASRequirements.sysml}

\clearpage
\subsection*{Conceptual Architecture Model}\label{app:arch-mod}
\subsubsection*{Conceptual Architecture Model with Focus on Components }
\begin{figure}[htbp]
  \centering
  \includegraphics[width=0.8\linewidth]{ressources/MAS/diagrams/MAS_architecture_complete_appendix.png}
  \caption{This figure shows the conceptual architecture model with a focus on the components of the system.}
  \label{fig:app-mas-arch-mod}
\end{figure}

\clearpage
\subsubsection*{Conceptual Architecture Model with External Environment}
\begin{figure}[htbp]
  \centering
  \includegraphics[width=0.9\linewidth]{ressources/MAS/diagrams/MAS_arch_and_environment1.png}
  \caption{This figure incorporates to the previous one the external environment of the system (1/2).}
  \label{fig:app-mas-arch-mod}
\end{figure}

\clearpage
\begin{figure}[htbp]
  \centering
  \includegraphics[width=0.9\linewidth]{ressources/MAS/diagrams/MAS_arch_and_environment2.png}
  \caption{This figure incomporates to the previous one the external environment of the system (2/2).}
  \label{fig:app-mas-arch-mod}
\end{figure}

\clearpage
\subsection{Requirements Traceability}\label{app:trace}
\begin{figure}[htbp]
  \centering
  \includegraphics[width=0.7\linewidth]{ressources/MAS/diagrams/MASTraceability1.png}
  \caption{Traceability overview. System elements satisfied consolidated requirements, and test cases verified a representative subset (1/3).}
  \label{fig:app-mas-traceability}
\end{figure}

\clearpage
\subsection{Requirements Traceability}\label{app:trace}
\begin{figure}[htbp]
  \centering
  \includegraphics[width=0.7\linewidth]{ressources/MAS/diagrams/MASTraceability2.png}
  \caption{Traceability overview. System elements satisfied consolidated requirements, and test cases verified a representative subset (2/3).}
  \label{fig:app-mas-traceability}
\end{figure}

\clearpage
\subsection{Requirements Traceability}\label{app:trace}
\begin{figure}[htbp]
  \centering
  \includegraphics[width=0.6\linewidth]{ressources/MAS/diagrams/MASTraceability3.png}
  \caption{Traceability overview. System elements satisfied consolidated requirements, and test cases verified a representative subset (3/3).}
  \label{fig:app-mas-traceability}
\end{figure}

\clearpage
\lstinputlisting[
  style=sysml,
  label={lst:mas-trace-src},
  numbers=left,
  inputencoding=utf8
]{./ressources/MAS/MASTraceability.sysml}

% --- AI Declaration ---

\clearpage

\begin{figure}[htbp]
  \centering
  \includegraphics[width=\linewidth]{ressources/AI-usage/KI-NE1.png}
\end{figure}
\begin{figure}[htbp]
  \centering
  \includegraphics[width=\linewidth]{ressources/AI-usage/KI-NE2.png}
\end{figure}
\begin{figure}[htbp]
  \centering
  \includegraphics[width=\linewidth]{ressources/AI-usage/KI-NE3.png}
\end{figure}
\begin{figure}[htbp]
  \centering
  \includegraphics[width=\linewidth]{ressources/AI-usage/KI-NE4.png}
\end{figure}